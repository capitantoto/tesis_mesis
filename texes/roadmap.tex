Para mi tesis quiero probar dos experimentos.

1. Cuan variables son muestras aleatorias de votantes en funcion del tamano de la muestra, manteniendo la representatividad a distintos niveles: mesa, circuito, comuna (seccion).

2. Cuan variables son lso resultados de la eleccion en funcion del orden en que se van cargando los telegramas.

Como lo realmente importante en una eleccion es saber quien se lleva cuantas bancas y no el \% de votos que uno saque, los resultados se mediran en funcion de en cuantas bancas totales varian con respecto al resultado que efectivamente se observo. Para esto voy a necesitar una calculadora D'Hont.

Para probar ambas hipotesis, trabajare unicamente con los datos de la eleccion de diputados de la capital federal. Los resultados deberian ser sencillamente reproducibles para senadores, y con ciertos retoques para otros distritos. Para trabajar con los datos de capital, necesito transformarlos a un formato accesible en ruby. Esto ya esta hecho.

Para probar la hipotesis 2., necesito escribir un pedazo de codigo que:

HECHO	a. Ordene al azar el array con los telegramas.
HECHO	b. Transforme el array en otro donde cada elemento i es la suma de todos los elementos hasa i inclusive.
HECHO	c. Normalice el array anterior para que cada uno de los elementos represente el procentaje correspondiente del total, entre 0 y 1.
HECHO	d. Calcule el ECM / dif por D'Hont del array anterior con la realidad a cada paso.
	e. Repetir el proceso anterior unas 10.000 veces y guardar los resultados.
	f. Plotear los resultados obtenidos.

Para probar la hipotesis 1, necesito escribir un pedazo de codigo que:

	a. Elija a que nivel de agregado 'agr' se quiere probar la hipotesis: mesa 'm', circuito 'c' o seccion 's'
	b. Elija que tama\~no de muestra 'n' se desea.
	c. Calcule cuantos votos por 'agr' se deben tomar para obtener el tamano de muestra 'n'
		i. Si se quiere obtener una muestra de 'n' votantes de un total de N, y cada agregado agr[i] tiene peso[i] votantes, habra que tomar n / N * peso[i] votos de cada agregado para la muestra general.
		ii. Para respetar dicha composicion, cuando se requiera tomar una cantidad no entera de votos X.YYY, se tomaran siempre al menos X votos, e X+1 votos con probabilidad 0.YYY.
	d. Tome dicha cantidad de votos sobre el agregado elegido, repetidas veces.
	e. Repita los pasos b-f de la hipotesis 2, pero para los muestreos generados en la hip 1.

Siendo las 16:30 del Martes 4/3, empiezo con la hipotesis 2.




Agrupaciones políticas 	Votos 	Estimación de cargos
PRO 	621.167 	34,46% 	5
UNEN 	581.096 	32,23% 	5
FPV 	389.128 	21,59% 	3
FIT 	101.862 	5,65% 	
AYL 	68.246 		3,79% 	
CP 	41.194 		2,28% 	
Afirm 	1.802.693 	93,21%
Blanco 	111.983 	5,79%
Nulos 	18.279 		0,95%
Rec&Imp	1.051 		0,05%	

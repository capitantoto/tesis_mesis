\documentclass[12pt, a4paper]{article}
\usepackage[spanish]{babel}
%\usepackage{a4wide}
\usepackage{amsmath}
\usepackage{amssymb}
\usepackage{amsthm}
\usepackage{color}
\usepackage{hyperref}
\usepackage{makeidx}
\usepackage{float}
\usepackage[margin=1.0in]{geometry}
\usepackage{caratula}
\usepackage{pdfpages}

\usepackage{listings}\usepackage{color}
\usepackage{textcomp}\definecolor{listinggray}{gray}{0.9}

\definecolor{lbcolor}{rgb}{0.9,0.9,0.9}
\usepackage{a4wide}

\lstset{
      language=C++,
      basicstyle=\small\sffamily,
      columns=fullflexible,
      upquote=true,
      extendedchars=true,
      texcl=true,
      mathescape=true,
      showspaces=false
                }
            


\makeindex

\begin{document}

\index{Caratula}
%Estos son los parametros para la caratula
\materia{M\'etodos Num\'ericos}
\submateria{Recuperatorio Trabajo Pr\'actico 2}
\titulo{Filtrado de se\~nales usando DCT}
\fecha{19 de Julio de 2013}
\integrante{Zajdband, Dan}{144/10}{dan.zajdband@gmail.com}
\integrante{Ferreria, Manuel}{199/10}{m.ferreria@gmail.com}
\integrante{Taravilse, Leopoldo}{464/08}{ltaravilse@gmail.com}
\resumen{Estudio comparativo de filtros de se\~nales mediante la utilizaci\'on
de la DCT. An\'alisis del efecto de diversos filtros para distintos casos de
ruido (Impulsivo y Blanco). Extensi\'on del problema a dos dimensiones. Estudio
del error de recuperaci\'on de la se\~nal original mediante la utilizaci\'on del
PSNR.}
\kwagregar{DCT, FFT, Fourier, Noise Reduction, Image analysis}

\maketitle

\pagebreak

\tableofcontents

\pagebreak

\index{Introducci\'on te\'orica}
\input{introduccion.tex}

\pagebreak

\index{Desarrollo}
\input{desarrollo.tex}

\pagebreak


\index{Resultados}
\input{resultados.tex}

\pagebreak


\index{Discusi\'on}
\input{discusion.tex}

\pagebreak

\input{conclusiones.tex}

\pagebreak

\input{apendices.tex}

\pagebreak

\input{referencias.tex}


\end{document}
